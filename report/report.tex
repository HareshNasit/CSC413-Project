\documentclass{article}

% if you need to pass options to natbib, use, e.g.:
%     \PassOptionsToPackage{numbers, compress}{natbib}
% before loading neurips_2021

% ready for submission
\usepackage[final]{neurips_2021}

% to compile a preprint version, e.g., for submission to arXiv, add add the
% [preprint] option:
%     \usepackage[preprint]{neurips_2021}

% to compile a camera-ready version, add the [final] option, e.g.:
%     \usepackage[final]{neurips_2021}

% to avoid loading the natbib package, add option nonatbib:
%    \usepackage[nonatbib]{neurips_2021}

\usepackage[utf8]{inputenc} % allow utf-8 input
\usepackage[T1]{fontenc}    % use 8-bit T1 fonts
\usepackage{hyperref}       % hyperlinks
\usepackage{url}            % simple URL typesetting
\usepackage{booktabs}       % professional-quality tables
\usepackage{amsfonts}       % blackboard math symbols
\usepackage{nicefrac}       % compact symbols for 1/2, etc.
\usepackage{microtype}      % microtypography
\usepackage{xcolor}         % colors
\usepackage{amsmath}
\usepackage{algorithm}      % algorithms
\usepackage{algpseudocode}  % algorithms
\usepackage[english]{babel} % citations
\usepackage{natbib}         % citations

\usepackage{dirtytalk}      % quotations
\bibliographystyle{abbrvnat}
\setcitestyle{numbers,open={[},close={]}} %Citation-related commands

\title{Single-Encoder Image-To-Image Translation}

% The \author macro works with any number of authors. There are two commands
% used to separate the names and addresses of multiple authors: \And and \AND.
%
% Using \And between authors leaves it to LaTeX to determine where to break the
% lines. Using \AND forces a line break at that point. So, if LaTeX puts 3 of 4
% authors names on the first line, and the last on the second line, try using
% \AND instead of \And before the third author name.

\author{%
  Adit Gajjar
  % \thanks{Use footnote for providing further information
  %   about author (webpage, alternative address)---\emph{not} for acknowledging
  %   funding agencies.}
  \\
  University of Toronto\\
  % Cranberry-Lemon University\\
  % Pittsburgh, PA 15213 \\
  % \texttt{hippo@cs.cranberry-lemon.edu} \\
  % examples of more authors
  \And
  Harsh Nasit \\
  University of Toronto\\
  % Address \\
  % \texttt{email} \\
  \And
  Filip Balucha \\
  University of Edinburgh\\
  % Address \\
  % \texttt{filip.balucha@mail.utoronto.ca} \\
}

\begin{document}

\maketitle

% TODO: remove line numbers?
\begin{abstract}
  Image-to-image translation is a computer vision problem where an image is mapped from one domain to another. In the supervised scenario, the dataset consists of perfectly-paired image data. However, such data is either hard to find or simply does not exist \citep{zhu}. Therefore, we focus on the unsupervised setting, where a joint distribution of images across two domains is learnt using the marginal distributions – the images in individual domains \citep{liu}. We propose a model that builds on the shared latent space assumption and uses domain-specific decoders. In addition, our approach readily extends to more than two different domains. The implementation is available at https://github.com/HareshNasit/CSC413-Project.
\end{abstract}

\section{Introduction}
At a high level, image-to-image translation aims to capture the characteristics of one image collection and determine how these could be translated into another image collection \citep{zhu}. Mathematically, the goal is to learn a mapping $G: X \rightarrow Y$, where X is the source and Y the target domain.

Image-to-image translation includes many computer vision problems, such as super-resolution, colorization and style transfer \citep{liu}. Style transfer has various applications in photo and video editing, customized gaming styles and commercial art. Transfering styles between images is a hard image processing task because it is difficult to represent semantic information such as image texture.

Many efforts were put into the supervised setting, where models learn image-to-image translation across two domains using perfectly-paired images \citep{zhu}. However, obtaining paired data is often difficult and expensive, and sometimes impossible \citep{zhu}. For example, to learn the translation from real-life photographs to images styled as Monet paintings, the supervised setting would require real-life photographs and their rendering by Monet – an infeasible task. Therefore, unsupervised algorithms that can map between domains without paired examples are desirable and shall be the focus of this paper.

From a probabilistic perspective, the aim is to learn a joint distribution of images between two different domains, where each domain consists of images from a marginal distribution \citep{liu}. However, there are infinitely many such joint distributions \citep{liu}. To address this problem, we can add more structure to the objective \citep{zhu} or make assumptions about the structure of the joint distribution \citep{liu}.

For example, the CycleGAN model resolves this by training two mappings $G: X \rightarrow Y$ and $F: Y \rightarrow X$, and updating the objective to enforce $F(G(x)) \approx x$ and $G(F(y)) \approx y$ \citep{zhu}. By contrast, \citet{liu} assume a shared latent space, where a pair of corresponding images from different domains is mapped to the same latent representation.

Inspired by the stable training and meaningful encodings characteristic of variational autoencoder (VAE) models, we continue in the latter vein. In particular, we propose a model that leverages the shared latent space assumption but is simpler than the model proposed by \citet{liu}. While our model does not outperform their model, we propose possible future directions that could bring our architecture to fruition.

\section{Related works}
Image-to-image translation is a well-studied problem. We provide a summary of papers that tackle the unsupervised setting. We focus on two well-known architectures – the popular Generative Adversarial Networks (GANs) and the Variational Autoencoders (VAEs), praised for learning useful representations that lend themselves to manipulation.

\textbf{GANs.} CycleGAN builds on the successful GAN architecture by introducing cycle-consistency. This constrains the image-to-image translation problem and encourages meaningful translations \citep{zhu}.

\textbf{VAEs.} Several methods use VAEs in the context of image-to-image translation or style transfer. ST-VAE performs multiple style transfer by projecting styles to a linear latent space and merging them using linear interpolation \citep{liu2}. \citet{kazemi} propose a framework that explicitly models two representations – one for image content and another for style. \citet{larsen} propose VAE-GAN, which combines a VAE and the discriminator from the GAN architecture. Finally, \citet{jha} apply the cycle-consistency idea to the VAE setting.

\textbf{VAEs and Shared latent representation.} Several research efforts establish a shared latent space. \citet{zhao} propose a mapping between the two latent spaces corresponding to the two domains. Perhaps closest to our approach, \citet{liu} share the latent space z between the two domains by sharing some weights of domain-specific encoders and decoders. Our simplified architecture achieves the same using a single, shared encoder.

\section{Method}
\subsection{Objective}
The objective adapts the vanilla VAE objective that consists of a reconstruction loss and a KL divergence term.
\subsubsection{Variational loss}
The variational loss term regularizes the latent space. It captures the KL divergence between the approximated posterior and the prior distribution. The posterior represents the \say{encoder} distribution, and the prior represents the family of distribution the encoder distribution should belong to. Similar to the original VAE paper, we assume both of these distributions are Gaussian \citep{kingma}. This allows us to express the variational loss term as
\begin{equation}
  \mathcal{L}_{\text{Variational}}
    = \frac{1}{\left|\mathcal{B}\right|} \sum_{i \in \mathcal{B}} \sum_{j=1}^J \left(
      1
      + \log\left(
        \left(
          \sigma_j^{(i)}
        \right)^2
      \right)
      - \left(\mu_j^{(i)}\right)^2 
      - \left(\sigma_j^{(i)}\right)^2 
    \right)
\end{equation}
where $\sigma$ and $\mu$ determine the approximated \say{encoder} distribution, $J$ is the latent dimension, and $i$ ranges over all samples in the batch $\mathcal{B}$.
\subsubsection{Reconstruction loss}
The original VAE paper uses a MSE reconstruction loss function \citep{kingma}. However, in the context of images, this loss function assumes that the impact of noise is independent of the local characteristics \citep{zhao2}. We therefore use a multiscale structural similarity index (MS-SSIM) term. MS-SSIM is a differentiable similarity metric that takes into account local features at various scale \citep{zhao2} and aims to produce visually-pleasing outputs. However, we found that using MS-SSIM term alone led to slow convergence. Therefore, similar to \citet{zhao2}, we instead augment the original MSE loss with MS-SSIM. We therefore express the reconstruction loss as
\begin{equation}
  \mathcal{L}_{\text{Reconstruction}}
    = \frac{1}{\left|\mathcal{B}\right|} \sum_{i \in \mathcal{B}}
    \text{MSE}\left(
      \mathbf{u}^{(i)},
      \mathbf{\tilde{u}}^{(i)}
    \right)
    + \text{MS-SSIM}\left(
      \mathbf{u}^{(i)},
      \mathbf{\tilde{u}}^{(i)}
    \right)
\end{equation}

\subsubsection{Full loss}
Finally, our full objective can be expresses as
\begin{equation}
  \mathcal{L}
  = \mathcal{L}_{\text{Variational}}
  + \alpha\cdot \mathcal{L}_{\text{Reconstruction}}
\end{equation}
where $\alpha$ is used to account for different learning rates between the two loss terms.
\subsection{Architecture}
\textbf{Pretraining – Encoder.} Our architecture combines the components of the standard convolutional VAE (Figure \ref{fig:pretrain}). Both domains share a single encoder $E$, which is trained in concord with a shared decoder $D$. Its purpose is to establish a shared latent space where, based on the shared latent space assumption, images with similar content from different domains will be mapped to a similar point.
% TODO: figure: Learning a shared latent representation for both image domains.
\begin{algorithm}[H]
  \caption{The pre-training procedure}\label{alg:pre-train}
  \begin{algorithmic}
    \State{// $X, Y \gets \text{loadDataset}()$}
    \State{$M \gets \text{new VAE}()$}
    
    \State{$E \gets \text{new Encoder}()$}
    \State{$M\text{.encoder} \gets E$}
    
    \State{$D \gets \text{new Decoder}()$}
    \State{$M\text{.decoder} \gets D$}
    
    \State{$M\text{.train}(X \cup Y)$}
    
  \end{algorithmic}
  \end{algorithm}

\textbf{Training – Domain-specific decoders.} Once the encoder is trained, the encoder weights are frozen to fix the mapping to a shared latent space. Then, a separate decoder is trained for each domain (Figure \ref{fig:train}). For example, if we wish to map between images of apples and oranges, $E$ would encode images of both, apples and oranges, and there would be a dedicated decoder for each apples and oranges, respectively. 
% TODO: figure: Learning the decoder D_X for domain X that uses embeddings from the shared latent space.
\begin{algorithm}[H]
  \caption{The training procedure}\label{alg:train}
  \begin{algorithmic}
    \State{// $X, Y \gets \text{loadDataset}()$}
    \State{$D_X \gets \text{new Decoder}()$}
    \State{$M\text{.decoder} = D_X$}
    
    \State{$M$.encoder.freeze()}
    
    \State{$M\text{.train}(X)$}
    \end{algorithmic}
\end{algorithm}

\textbf{Evaluation – Style transfer.} To perform style transfer, we simply combine the encoder $E$ and a domain-specific decoder $D_X$ or $D_Y$. For example, to obtain a mapping from domain $Y$ to $X$, it suffices to use the encoder E and decoder $D_X$ (Figure \ref{fig:eval}).
% TODO: figure: Performing style transfer from domain Y to X using the shared encoder E and domain-specific decoder D_X.
\begin{algorithm}[H]
  \caption{The evaluation procedure}\label{alg:eval}
  \begin{algorithmic}
    \State{$y \gets SampleFrom(Y)$}
    \State{$M\text{.predict}(y)$}
  \end{algorithmic}
\end{algorithm}

\subsection{Training details}
\subsubsection{Layer dimensions}
The shared encoder E and the domain-specific decoders $D_x$ and $D_y$ have the same size. Both the encoder and decoders are 5-layer deep with the following hidden sizes: 32, 64, 128, 256 and 512. The latent representation was a 256-dimensional vector. Each encoder and decoder layer included, in sequence, a convolution or transposed convolution, batch normalization, and the Leaky ReLU activation function. We use the Adam optimizer with a batch size of 64 and $\gamma = 0.95$. Hyperparameters used in the appendix.

\subsubsection{Datasets}
We trained the model on the \texttt{apple2orange}, \texttt{summer2winter\_yosemite} and \texttt{horse2zebra} datasets. The images were split into domain-specific (i.e. apple and orange) datasets. The domain-specific datasets were used to train domain-specific decoders, and the combined domain dataset was used to train the shared encoder. The images used were of dimensions $256\times256$.
\subsubsection{Training}
The model was trained on Google Colab using a single NVIDIA Tesla K10 GPU. The shared encoder and domain-specific decoders were trained for a maximum of 70 epochs, respectively. The training duration was approximately 80 minutes.

\section{Experiments}
We performed image translation experiments using three different datasets. The hyperparameters used to train on all the datasets can be found in Table 1. The model in general results in translations that have some characteristics of the target domain but the output images still have blurry edges and lacks clarity.
\subsection{\texttt{apple2orange}}
Our model's performance on \texttt{apple2orange} dataset was promising as it was able to learn the shape and color of the fruits really well. For instance, in the below translation of an orange to an apple, the translation removed the leaf and the stem of the orange. The translations of orange to apple visually looks better than apple to orange.
% TODO: figure

\subsection{\texttt{summer2winter\_yosemite}}
Our model’s performance on \texttt{summer2winter\_yosemite} was not as good as \texttt{apple2orange} dataset. It was able to learn the color of the sky and the trees. The translations of summer images to winter resulted in output images that had gray skies and black trees. For winter images as input, the output translations had blue colored skies and green trees. Visually we can tell the difference between the translations of the two seasons.
% TODO: figure

\subsection{\texttt{horse2zebra}}
Our model performed the worst on the \texttt{horse2zebra} dataset. Visually we cannot tell the difference between the translations of the two animals. The translations just show the shadow of the animals. Since the horse dataset consists of horses of multiple colors, it is possible the model was not able to learn physical features of the two animals.

% TODO: figure

% \subsection{Evaluation metrics}

% \subsection{Limitations}

\section{Conclusion and future work}
Our VAE architecture performs worse than CycleGAN or the related architecture proposed by \citet{liu}. While the output content (e.g. the rough horse and zebra silhouettes) is maintained, the output quality is subpar. Our domain-specific decoder was likely limited by the quality of the latent representation produced by the shared encoder.

A more complex encoder, such as the one proposed by \citet{liu}, could learn a better representation of the domains and thus produce better translations. Another way to produce higher-quality images would be to replace our standard VAE architecture with a hierarchical VAE or VQ-VAE which has been shown to produce higher quality reconstructions. In addition, a more-informed loss function could improve quality. For instance, similar to \citet{liu}, a loss term based on a discriminator network could be used for translated images.


% Note: citations work like this:
% \cite{liu}
% \citet{liu}
% \citep{liu}

\bibliography{sample}

% Template:
\newpage
\section{Submission of papers to NeurIPS 2021}

Please read the instructions below carefully and follow them faithfully.

\subsection{Style}

Papers to be submitted to NeurIPS 2021 must be prepared according to the
instructions presented here. Papers may only be up to {\bf nine} pages long,
including figures. Additional pages \emph{containing only acknowledgments and
references} are allowed. Papers that exceed the page limit will not be
reviewed, or in any other way considered for presentation at the conference.

The margins in 2021 are the same as those in 2007, which allow for $\sim$$15\%$
more words in the paper compared to earlier years.

Authors are required to use the NeurIPS \LaTeX{} style files obtainable at the
NeurIPS website as indicated below. Please make sure you use the current files
and not previous versions. Tweaking the style files may be grounds for
rejection.

\subsection{Retrieval of style files}

The style files for NeurIPS and other conference information are available on
the World Wide Web at
\begin{center}
  \url{http://www.neurips.cc/}
\end{center}
The file \verb+neurips_2021.pdf+ contains these instructions and illustrates the
various formatting requirements your NeurIPS paper must satisfy.

The only supported style file for NeurIPS 2021 is \verb+neurips_2021.sty+,
rewritten for \LaTeXe{}.  \textbf{Previous style files for \LaTeX{} 2.09,
  Microsoft Word, and RTF are no longer supported!}

The \LaTeX{} style file contains three optional arguments: \verb+final+, which
creates a camera-ready copy, \verb+preprint+, which creates a preprint for
submission to, e.g., arXiv, and \verb+nonatbib+, which will not load the
\verb+natbib+ package for you in case of package clash.

\paragraph{Preprint option}
If you wish to post a preprint of your work online, e.g., on arXiv, using the
NeurIPS style, please use the \verb+preprint+ option. This will create a
nonanonymized version of your work with the text ``Preprint. Work in progress.''
in the footer. This version may be distributed as you see fit. Please \textbf{do
  not} use the \verb+final+ option, which should \textbf{only} be used for
papers accepted to NeurIPS.

At submission time, please omit the \verb+final+ and \verb+preprint+
options. This will anonymize your submission and add line numbers to aid
review. Please do \emph{not} refer to these line numbers in your paper as they
will be removed during generation of camera-ready copies.

The file \verb+neurips_2021.tex+ may be used as a ``shell'' for writing your
paper. All you have to do is replace the author, title, abstract, and text of
the paper with your own.

The formatting instructions contained in these style files are summarized in
Sections \ref{gen_inst}, \ref{headings}, and \ref{others} below.

\section{General formatting instructions}
\label{gen_inst}

The text must be confined within a rectangle 5.5~inches (33~picas) wide and
9~inches (54~picas) long. The left margin is 1.5~inch (9~picas).  Use 10~point
type with a vertical spacing (leading) of 11~points.  Times New Roman is the
preferred typeface throughout, and will be selected for you by default.
Paragraphs are separated by \nicefrac{1}{2}~line space (5.5 points), with no
indentation.

The paper title should be 17~point, initial caps/lower case, bold, centered
between two horizontal rules. The top rule should be 4~points thick and the
bottom rule should be 1~point thick. Allow \nicefrac{1}{4}~inch space above and
below the title to rules. All pages should start at 1~inch (6~picas) from the
top of the page.

For the final version, authors' names are set in boldface, and each name is
centered above the corresponding address. The lead author's name is to be listed
first (left-most), and the co-authors' names (if different address) are set to
follow. If there is only one co-author, list both author and co-author side by
side.

Please pay special attention to the instructions in Section \ref{others}
regarding figures, tables, acknowledgments, and references.

\section{Headings: first level}
\label{headings}

All headings should be lower case (except for first word and proper nouns),
flush left, and bold.

First-level headings should be in 12-point type.

\subsection{Headings: second level}

Second-level headings should be in 10-point type.

\subsubsection{Headings: third level}

Third-level headings should be in 10-point type.

\paragraph{Paragraphs}

There is also a \verb+\paragraph+ command available, which sets the heading in
bold, flush left, and inline with the text, with the heading followed by 1\,em
of space.

\section{Citations, figures, tables, references}
\label{others}

These instructions apply to everyone.

\subsection{Citations within the text}

The \verb+natbib+ package will be loaded for you by default.  Citations may be
author/year or numeric, as long as you maintain internal consistency.  As to the
format of the references themselves, any style is acceptable as long as it is
used consistently.

The documentation for \verb+natbib+ may be found at
\begin{center}
  \url{http://mirrors.ctan.org/macros/latex/contrib/natbib/natnotes.pdf}
\end{center}
Of note is the command \verb+\citet+, which produces citations appropriate for
use in inline text.  For example,
\begin{verbatim}
   \citet{hasselmo} investigated\dots
\end{verbatim}
produces
\begin{quote}
  Hasselmo, et al.\ (1995) investigated\dots
\end{quote}

If you wish to load the \verb+natbib+ package with options, you may add the
following before loading the \verb+neurips_2021+ package:
\begin{verbatim}
   \PassOptionsToPackage{options}{natbib}
\end{verbatim}

If \verb+natbib+ clashes with another package you load, you can add the optional
argument \verb+nonatbib+ when loading the style file:
\begin{verbatim}
   \usepackage[nonatbib]{neurips_2021}
\end{verbatim}

As submission is double blind, refer to your own published work in the third
person. That is, use ``In the previous work of Jones et al.\ [4],'' not ``In our
previous work [4].'' If you cite your other papers that are not widely available
(e.g., a journal paper under review), use anonymous author names in the
citation, e.g., an author of the form ``A.\ Anonymous.''

\subsection{Footnotes}

Footnotes should be used sparingly.  If you do require a footnote, indicate
footnotes with a number\footnote{Sample of the first footnote.} in the
text. Place the footnotes at the bottom of the page on which they appear.
Precede the footnote with a horizontal rule of 2~inches (12~picas).

Note that footnotes are properly typeset \emph{after} punctuation
marks.\footnote{As in this example.}

\subsection{Figures}

\begin{figure}
  \centering
  \fbox{\rule[-.5cm]{0cm}{4cm} \rule[-.5cm]{4cm}{0cm}}
  \caption{Sample figure caption.}
\end{figure}

All artwork must be neat, clean, and legible. Lines should be dark enough for
purposes of reproduction. The figure number and caption always appear after the
figure. Place one line space before the figure caption and one line space after
the figure. The figure caption should be lower case (except for first word and
proper nouns); figures are numbered consecutively.

You may use color figures.  However, it is best for the figure captions and the
paper body to be legible if the paper is printed in either black/white or in
color.

\subsection{Tables}

All tables must be centered, neat, clean and legible.  The table number and
title always appear before the table.  See Table~\ref{sample-table}.

Place one line space before the table title, one line space after the
table title, and one line space after the table. The table title must
be lower case (except for first word and proper nouns); tables are
numbered consecutively.

Note that publication-quality tables \emph{do not contain vertical rules.} We
strongly suggest the use of the \verb+booktabs+ package, which allows for
typesetting high-quality, professional tables:
\begin{center}
  \url{https://www.ctan.org/pkg/booktabs}
\end{center}
This package was used to typeset Table~\ref{sample-table}.

\begin{table}
  \caption{Sample table title}
  \label{sample-table}
  \centering
  \begin{tabular}{lll}
    \toprule
    \multicolumn{2}{c}{Part}                   \\
    \cmidrule(r){1-2}
    Name     & Description     & Size ($\mu$m) \\
    \midrule
    Dendrite & Input terminal  & $\sim$100     \\
    Axon     & Output terminal & $\sim$10      \\
    Soma     & Cell body       & up to $10^6$  \\
    \bottomrule
  \end{tabular}
\end{table}

\section{Final instructions}

Do not change any aspects of the formatting parameters in the style files.  In
particular, do not modify the width or length of the rectangle the text should
fit into, and do not change font sizes (except perhaps in the
\textbf{References} section; see below). Please note that pages should be
numbered.

\section{Preparing PDF files}

Please prepare submission files with paper size ``US Letter,'' and not, for
example, ``A4.''

Fonts were the main cause of problems in the past years. Your PDF file must only
contain Type 1 or Embedded TrueType fonts. Here are a few instructions to
achieve this.

\begin{itemize}

\item You should directly generate PDF files using \verb+pdflatex+.

\item You can check which fonts a PDF files uses.  In Acrobat Reader, select the
  menu Files$>$Document Properties$>$Fonts and select Show All Fonts. You can
  also use the program \verb+pdffonts+ which comes with \verb+xpdf+ and is
  available out-of-the-box on most Linux machines.

\item The IEEE has recommendations for generating PDF files whose fonts are also
  acceptable for NeurIPS. Please see
  \url{http://www.emfield.org/icuwb2010/downloads/IEEE-PDF-SpecV32.pdf}

\item \verb+xfig+ "patterned" shapes are implemented with bitmap fonts.  Use
  "solid" shapes instead.

\item The \verb+\bbold+ package almost always uses bitmap fonts.  You should use
  the equivalent AMS Fonts:
\begin{verbatim}
   \usepackage{amsfonts}
\end{verbatim}
followed by, e.g., \verb+\mathbb{R}+, \verb+\mathbb{N}+, or \verb+\mathbb{C}+
for $\mathbb{R}$, $\mathbb{N}$ or $\mathbb{C}$.  You can also use the following
workaround for reals, natural and complex:
\begin{verbatim}
   \newcommand{\RR}{I\!\!R} %real numbers
   \newcommand{\Nat}{I\!\!N} %natural numbers
   \newcommand{\CC}{I\!\!\!\!C} %complex numbers
\end{verbatim}
Note that \verb+amsfonts+ is automatically loaded by the \verb+amssymb+ package.

\end{itemize}

If your file contains type 3 fonts or non embedded TrueType fonts, we will ask
you to fix it.

\subsection{Margins in \LaTeX{}}

Most of the margin problems come from figures positioned by hand using
\verb+\special+ or other commands. We suggest using the command
\verb+\includegraphics+ from the \verb+graphicx+ package. Always specify the
figure width as a multiple of the line width as in the example below:
\begin{verbatim}
   \usepackage[pdftex]{graphicx} ...
   \includegraphics[width=0.8\linewidth]{myfile.pdf}
\end{verbatim}
See Section 4.4 in the graphics bundle documentation
(\url{http://mirrors.ctan.org/macros/latex/required/graphics/grfguide.pdf})

A number of width problems arise when \LaTeX{} cannot properly hyphenate a
line. Please give LaTeX hyphenation hints using the \verb+\-+ command when
necessary.

\begin{ack}
Use unnumbered first level headings for the acknowledgments. All acknowledgments
go at the end of the paper before the list of references. Moreover, you are required to declare
funding (financial activities supporting the submitted work) and competing interests (related financial activities outside the submitted work).
More information about this disclosure can be found at: \url{https://neurips.cc/Conferences/2021/PaperInformation/FundingDisclosure}.

Do {\bf not} include this section in the anonymized submission, only in the final paper. You can use the \texttt{ack} environment provided in the style file to autmoatically hide this section in the anonymized submission.
\end{ack}

\section*{References}

References follow the acknowledgments. Use unnumbered first-level heading for
the references. Any choice of citation style is acceptable as long as you are
consistent. It is permissible to reduce the font size to \verb+small+ (9 point)
when listing the references.
Note that the Reference section does not count towards the page limit.
\medskip

{
\small

[1] Alexander, J.A.\ \& Mozer, M.C.\ (1995) Template-based algorithms for
connectionist rule extraction. In G.\ Tesauro, D.S.\ Touretzky and T.K.\ Leen
(eds.), {\it Advances in Neural Information Processing Systems 7},
pp.\ 609--616. Cambridge, MA: MIT Press.

[2] Bower, J.M.\ \& Beeman, D.\ (1995) {\it The Book of GENESIS: Exploring
  Realistic Neural Models with the GEneral NEural SImulation System.}  New York:
TELOS/Springer--Verlag.

[3] Hasselmo, M.E., Schnell, E.\ \& Barkai, E.\ (1995) Dynamics of learning and
recall at excitatory recurrent synapses and cholinergic modulation in rat
hippocampal region CA3. {\it Journal of Neuroscience} {\bf 15}(7):5249-5262.
}

%%%%%%%%%%%%%%%%%%%%%%%%%%%%%%%%%%%%%%%%%%%%%%%%%%%%%%%%%%%%
\section*{Checklist}

%%% BEGIN INSTRUCTIONS %%%
The checklist follows the references.  Please
read the checklist guidelines carefully for information on how to answer these
questions.  For each question, change the default \answerTODO{} to \answerYes{},
\answerNo{}, or \answerNA{}.  You are strongly encouraged to include a {\bf
justification to your answer}, either by referencing the appropriate section of
your paper or providing a brief inline description.  For example:
\begin{itemize}
  \item Did you include the license to the code and datasets? \answerYes{See Section~\ref{gen_inst}.}
  \item Did you include the license to the code and datasets? \answerNo{The code and the data are proprietary.}
  \item Did you include the license to the code and datasets? \answerNA{}
\end{itemize}
Please do not modify the questions and only use the provided macros for your
answers.  Note that the Checklist section does not count towards the page
limit.  In your paper, please delete this instructions block and only keep the
Checklist section heading above along with the questions/answers below.
%%% END INSTRUCTIONS %%%

\begin{enumerate}

\item For all authors...
\begin{enumerate}
  \item Do the main claims made in the abstract and introduction accurately reflect the paper's contributions and scope?
    \answerTODO{}
  \item Did you describe the limitations of your work?
    \answerTODO{}
  \item Did you discuss any potential negative societal impacts of your work?
    \answerTODO{}
  \item Have you read the ethics review guidelines and ensured that your paper conforms to them?
    \answerTODO{}
\end{enumerate}

\item If you are including theoretical results...
\begin{enumerate}
  \item Did you state the full set of assumptions of all theoretical results?
    \answerTODO{}
	\item Did you include complete proofs of all theoretical results?
    \answerTODO{}
\end{enumerate}

\item If you ran experiments...
\begin{enumerate}
  \item Did you include the code, data, and instructions needed to reproduce the main experimental results (either in the supplemental material or as a URL)?
    \answerTODO{}
  \item Did you specify all the training details (e.g., data splits, hyperparameters, how they were chosen)?
    \answerTODO{}
	\item Did you report error bars (e.g., with respect to the random seed after running experiments multiple times)?
    \answerTODO{}
	\item Did you include the total amount of compute and the type of resources used (e.g., type of GPUs, internal cluster, or cloud provider)?
    \answerTODO{}
\end{enumerate}

\item If you are using existing assets (e.g., code, data, models) or curating/releasing new assets...
\begin{enumerate}
  \item If your work uses existing assets, did you cite the creators?
    \answerTODO{}
  \item Did you mention the license of the assets?
    \answerTODO{}
  \item Did you include any new assets either in the supplemental material or as a URL?
    \answerTODO{}
  \item Did you discuss whether and how consent was obtained from people whose data you're using/curating?
    \answerTODO{}
  \item Did you discuss whether the data you are using/curating contains personally identifiable information or offensive content?
    \answerTODO{}
\end{enumerate}

\item If you used crowdsourcing or conducted research with human subjects...
\begin{enumerate}
  \item Did you include the full text of instructions given to participants and screenshots, if applicable?
    \answerTODO{}
  \item Did you describe any potential participant risks, with links to Institutional Review Board (IRB) approvals, if applicable?
    \answerTODO{}
  \item Did you include the estimated hourly wage paid to participants and the total amount spent on participant compensation?
    \answerTODO{}
\end{enumerate}

\end{enumerate}

%%%%%%%%%%%%%%%%%%%%%%%%%%%%%%%%%%%%%%%%%%%%%%%%%%%%%%%%%%%%

\appendix

\section{Appendix}

Optionally include extra information (complete proofs, additional experiments and plots) in the appendix.
This section will often be part of the supplemental material.

\end{document}
